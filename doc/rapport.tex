\documentclass[12pt]{article}
\usepackage[utf8]{inputenc}
\usepackage[T1]{fontenc}
\usepackage[french]{babel}
\usepackage{amsmath,amsfonts,amssymb}
\usepackage{fullpage}
\usepackage{graphicx}
\usepackage{hyperref}


\title{Interprète Lisp en C++}
\author{Aurèle Barrière \& Jérémy Thibault}

\begin{document}
\maketitle
\tableofcontents
\newpage

\section{Partie obligatoire : interprète à liaison dynamique}

\subsection{Exceptions}
La première amélioration de notre toplevel fut de rattraper toutes les exceptions lancées par le programme. En effet, il ne fallait pas que l'interprète s'arrête si l'utilisateur fait une erreur : nous voulions qu'il lui indique le type d'erreur, l'objet concerné et qu'il continue à s'éxécuter.

Ainsi, nous avons créé un module \texttt{exceptions} contenant tous les types d'exceptions (des classes différentes) : exceptions liées à l'absence d'arguments, à un problème de typage, à un manque de liaisons dans l'environnement etc...

Ensuite, dans l'exécution du toplevel, on rattrape ces exceptions et on affiche les messages d'erreurs correspondant.

\subsection{Organisation du toplevel}

Dans le code initial, le fichier \texttt{main} contenait le toplevel et l'appel au parseur mélangés. Nous avons séparés ça.

Nous avons ainsi créé un module \texttt{toplevel} contenant une fonction \texttt{toplevel()} qui sera appelée dans le \texttt{main}. Ce module fait appel aux fonctions de bison pour parser un fichier donné.

\subsection{Subroutines}

Nous avons également eu besoin de rajouter des subroutines : au moins $-$ et $=$ étaient nécessaires pour implémenter des fonctions récursives.

Nous en avons profités pour séparer toutes les subroutines de l'évaluation dans un module distinct.

\subsection{La directive setq}

Pour ajouter des liaisons dans l'environnement courant, nous avons du ajouter un cas particulier à l'évaluation : \texttt{setq}. 

Lorsque l'utilisateur utilise cette commande, on évalue le deuxième argument, et on crée la liaison entre le premier argument et l'objet évalué.

%à rajouter : traitement du cas lambda a l'évaluation

\subsection{Mode verbeux et affichage d'environnement}

Nous avons également laissé à l'utilisateur la possibilité d'activer le mode verbeux (qui indique chaque appel d'évaluation ou d'application de fonction) avec un argument optionnel.

Une commande est également disponible pour afficher l'environnement courant.


\section{Partie optionnelle : extensions}

Nous avons choisi d'étudier la gestion mémoire de notre interpréteur.

\subsection{Allocation de cellules}

\subsection{Garbage collector}

\subsection{Mise en oeuvre et tests}

\end{document}
